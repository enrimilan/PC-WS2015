Es werden hier unsere 3 Implementierungen vorgestellt. Die Programme im Verzeichnis src enthalten alle jeweils eine main-Methode, die die Implementierung aufruft. Diese sind unter src/merge zu finden. Um das ganze zu kompilieren reicht es $make$ aufzurufen. 

\subsection{OpenMP}
Wie man dem Code entnehmen kann, sind $start$ und $end$ die Grenzen für jeden Durchlauf, also von wo bis wo jeder Prozess in das resultierende Array schreiben soll. Die Grenzen für die Input-Arrays (aus denen gelesen wird) werden durch die $coranks$-Implementierung, die wir vorgestellt haben, bestimmt.

Mithilfe der Konstrukte von OpenMP kann erreicht werden, dass jeder Durchlauf von einem eigenen Prozessor durchgeführt wird. 
\begin{verbatim}
#pragma omp parallel
{
    #pragma omp for
    for (int i=0; i<p; i++) {
        int startA, endA, startB, endB;
        long start = (long)i*(long)(lenA+lenB)/(long)p;
        long end = (long)(i+1)*(long)(lenA+lenB)/(long)p;
        coranks(start, A, lenA, &startA, B, lenB, &startB);
        coranks(end, A, lenA, &endA, B, lenB, &endB);
        sequential_merge(A+startA, endA-startA, B+startB, 
             endB-startB, result+start);
    }
}
\end{verbatim}
Die Durchläufe finden also parallel statt. Die Prozesse/Threads arbeiten dabei unabhängig von einander, kommen sich also nicht in die Quere. Da die Grenzen eindeutig bestimmt sind, können mehrere Prozesse gleichzeitig das resultierende Array mit den gemergten Daten befüllen, indem sie die sequentielle Implementierung aufrufen.

\subsection{Cilk}
Der rekursive Algorithmus hier funktioniert nach dem \emph{divide and conquer} Prinzip. Das resultierende Array wird dabei immer in zwei Hälften aufgeteilt und für jede Hälfte ein neuer Task gespawned. Das wird so lang gemacht, bis das Array von $start$ bis $end$ nicht zu groß für ein Paket ist. $unit$ bezeichnet dabei die Maximalgröße eines Pakets, das ein Thread selbst merged.
\begin{verbatim}
if (end-start <= unit) {
      int startA, endA, startB, endB;
     // Merge data
     coranks(start, A, lenA, &startA, B, lenB, &startB);
     coranks(end, A, lenA, &endA, B, lenB, &endB);
     sequential_merge(A+startA, endA-startA, B+startB, 
           endB-startB, result+start);
}
else {
      // Distribute work to other threads
      int middle = start + (end-start)/2;
      cilk_spawn cilk_merge_part(start, middle, unit, A, lenA, B, lenB, result);
      cilk_spawn cilk_merge_part(middle, end, unit, A, lenA, B, lenB, result);
}
\end{verbatim}

\subsection{MPI}
Die MPI-Implementierung ist ein bisschen anders, da hier keine Shared-Memory verwendet wird. Stattdessen kommunizieren die Prozesse miteinander. Diese Implementierung baut auf einem Paper~\cite{mpi} auf. Dabei wurden folgende Annahmen getroffen:
\begin{itemize}
\item die Anzahl der Prozessoren $p$ teilt die Summe der Längen beider Arrays bzw die Länge des resultierenden Arrays
\item die Input-Arrays haben beide gleiche Längen
\end{itemize}
Wie die namen vermuten lassen, ist \verb|mpi_merge_master()| die Methode die im Root-Prozess aufgerufen wird und \verb|mpi_merge_slave()| die die von den restlichen Prozessen verwendet wird. Als allererstes müssen die Daten an die anderen Prozesse verteilt werden. Das passiert in \verb|distribute_data()|. Durch die getroffenen Annahmen war die Verteilung der Daten nicht sehr komplex. Da die Prozesse eventuell miteinander kommunizieren müssen um die co-ranks zu berechnen, wurden communication windows verwendet um die Daten offenzulegen.
\begin{verbatim}
//expose the data
MPI_Win_create(localdata_A, local_array_A_length, 
      sizeof(data_t), MPI_INFO_NULL, MPI_COMM_WORLD, win_A);
MPI_Win_create(localdata_B, local_array_B_length, 
      sizeof(data_t), MPI_INFO_NULL, MPI_COMM_WORLD, win_B);
\end{verbatim}
Außerdem war es nötig, eine neue Implementierung für den Algorithmus der die co-ranks ausrechnet, zu schreiben. Der Zugriff auf die Arrays wird dabei durch die Methode \verb|get_data()| abstrahiert. 
\begin{verbatim}
data_t get_data(data_t *chunk, long start, long end, long len, MPI_Win win){
      long elements_per_process = len / world_size;
      int target_rank = start / elements_per_process;
      long local_index = start - (target_rank * elements_per_process);
      long count = end - start + 1;
      long chunk_position = 0;
      long remaining;
      while(count>0){
           remaining = elements_per_process - local_index;
           long to_copy = remaining > count ? count : remaining;
           MPI_Win_lock(MPI_LOCK_SHARED, target_rank, MPI_MODE_NOCHECK, win);
           MPI_Get(chunk+chunk_position, to_copy, MPI_LONG, 
                  target_rank, local_index, to_copy, MPI_LONG, win);
           MPI_Win_unlock(target_rank, win);
           count = count - remaining;
           local_index = 0;
           target_rank++;
           chunk_position = chunk_position + remaining;
      }
      return chunk[0];
}
\end{verbatim}
Die Synchronisation erfolgt über shared locks. Durch \verb| MPI_MODE_NOCHECK| gibt man an, dass Zugriffe gleichzeitig erfolgen können, da man die Daten nur liest. Am Ende werden die verschmelzten Daten wieder abgesammelt.