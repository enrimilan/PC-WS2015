Wir haben eine Testumgebung entwickelt, die es uns erlaubt automatisiert zu testen sowie Ausführungszeiten zu messen und anschließend zu plotten.
In diesem Abschnitt wird ihr grundlegender Aufbau näher erläutert.
Konkrete Details zu den unterschiedlichen Testszenarien sind in Abschnitt 5 zu finden.

Die Testumgebung besteht aus einer Reihe von C-Programmen und Shell Skripts, sowie einem \texttt{gnuplot} Skript, das für die grafische Darstellung der Benchmarks zuständig ist.
Für jede parallele Implementierung existiert ein C-Programm, das sich um die Ausführung der einzelnen Testfälle kümmert.
Als einzigen Parameter verlangt es die Anzahl der Worker, die für das parallele Mergen eingesetzt werden sollen.

Die Testprogramme führen nun für eine Reihe von Testfällen sowohl die sequentielle, als auch die jeweilige parallele Implementierung aus.
In MPI kümmert sich ausschließlich der \emph{Master}prozess (Prozess mit dem Rank 0) um die Ausführung der Tests.
Alle anderen Prozesse sind \emph{Slaves} und warten nach dem Start direkt auf die Verteilung der zu mergenden Arrays.
Um rauscharme Ergebnisse zu erhalten, wird jeder Testfall 100 mal ausgeführt.
In jeder Ausführung misst die Testumgebung mit, wie lange die jeweilige Implementierung für das Mergen der Daten gebraucht hat.
Anschließend werden die kleinsten und mittleren Ausführungszeiten der parallelen Implementierungen zusammen mit den erhaltenen Speedups (gruppiert nach Testfall und Implementierung) in Log-Dateien geschrieben.

Um die Korrektheit der implementierten Algorithmen zu gewährleisten wird bei jedem Testdurchlauf das Ergebnisse der sequentiellen Implementierung mit jenem der parallelen verglichen.
Kommt es hier zu einer Diskrepanz, wird der entsprechende Testfall in einem Error-Log festgehalten.
Da es während der Ausführung unserer Tests zu keinerlei Fehlern kam, gehen wir davon aus, dass die parallelen Implementierungen korrekt umgesetzt wurden.

Ferner existieren drei Shell Skripts, die die Ausführung der Tests auf den Rechnerclustern, sowie die Generierung der Plots automatisieren.
Das Skript \texttt{test.sh} lädt den Quellcode sowie je ein weiteres Shell Skript auf die beiden Cluster \emph{Saturn} und \emph{Jupiter} hoch.
Anschließend ruft es die beiden hochgeladenen Skripts auf, welche die Ausführung der Tests auf den jeweiligen Maschinen koordinieren.
Jedes der genannten Testprogramme wird nun von dem zuständigen Skript mit unterschiedlichen Anzahlen von Workern aufgerufen.
Im Abschluss lädt \texttt{test.sh} die erstellten Benchmarks herunter und stellt diese mit gnuplot grafisch dar.